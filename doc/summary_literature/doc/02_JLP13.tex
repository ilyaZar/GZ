\section{Rising Income Inequality:Technology, or Trade and Financial Globalization(F. Jaumotte, S. Lall, and C. Papageorgiou - 2013)}
\subsection{Objective of the Study}
The paper examines the relationship between the rapid pace of trade and financial globalization and the rise in income inequality observed in most countries over the past two decades. In particular, the effects of trade, financial globalization and technology on income inequality are studied.
\begin{itemize}
\item developed and devoloping countries as cross section (while existing papers mostly address within country experience for the specific country being studied)
\item identify the separate effects of both key dimensions of globalization: greater trade openess and greater financial openess (while exisiting literature has mainly focused primarily on trade with limited attention to financial globalization)
\item subcomponents of trade and financial globalization are analysed, see data section below $\Rightarrow$ it is expected that different subcomponents of globalization affect inequality in different ways.  examples are: exports of manufacturing vs. agricultural goods, portfolio debt and FDI
\item assembling of new data set on income inequality (using World Bank Povcal and Luxemburg income study databases) that produces higher methodoligcal consistency in survey-based ineq. measures across countries over time. as a result, inequality facsts across a large number of countries can be more accurately and comprohensively documented
\item interestingly, the effects on the bottom four quintiles are qualitatively similar and in the opposite direction from that on the richest quintile: export growth is associated with a rise in the income shares of the bottom four quintiles and a decrease
in the share of the richest quintile and, in contrast, financial globalization and technological progress are shown to benefit mainly the richest 20 percentof the population
\end{itemize}
\subsection{Method (Economic)}
There is no theoretical model but some economic intuition for the model structure can be provided:\\
To gain further insight into the impact of globalization on inequality, we also estimated our model using the income shares of the five quintiles of the population as dependent variables. 
\subsection{Method (Econoemtric)}
\subsection{Data}
Newly compiled panel of 51 countries over a 23-year period from 1981 to 2003:
\begin{itemize}
\item subcomponents of trade and financial globalization e.g. exports of manufacturing vs. agriculture, and portfolio debt and equity flows vs FDI
\end{itemize}
\subsection{Conclusion/Results/Main findings}
The paper reports estimates that support a greater impact of technological progress than globalization on inequality. The limited overall impact of globalization reflects two offsetting tendencies: whereas trade globalization is associated with a reduction in inequality, financial globalization - and foreign direct investment in particular - is associated with an
increase in inequality. More specifically:
\begin{itemize}
\item income inequality has risen in most countries and regions over past two decades
\item at the same time avg. real income of the poorest segements of the population have increased across all regions and income groups $\Rightarrow$ suggests that inequality has increased in the upper parts of the distribution in most countries, a fact consistent with recent evidence in the United States and the United Kingdom
\item trade liberalization and export growth are found to be associated with lower inequality
\item increased financial openness is associated with higher inequality
\item however, the combined contribution of trade and financial openness to rising inequality has been much lower than that of technological change, both at a global level and
especially markedly in developing countries
\item the spread of technology is, of course, itself related to increased globalization, but technological progress is nevertheless seen to have a separately identifiable effect on inequality
\item The disequalizing impact of financial openness - mainly felt through FDI - and technological progress both appear to be working by increasing the premium on higher skills and possibly higher returns to capital, rather than limiting opportunities for economic advancement
\item consistent with this, increased access to education is associated with more equal income distributions on average
\end{itemize}
\subsection{Remarks}
Technological progress and globalization are widely regarded as the main drivers of recent economic growth.
\begin{itemize}
\item[$\hookrightarrow$] technological progress: development and spread of new ideas and methods that enhance productivity and efficiency
\item[$\hookrightarrow$] globalization: catalyst of technology that facilitates the diffusion of ideas and methods
\item[$\hookrightarrow$] increase in globalization e.g. seen via  openess to trade (dismantling of trade barriers) or easing restricitons on FDI helps technology diffusion (world bank report) even though technological innovations occur only in a handful of advanced economies
\end{itemize}
However, the distributional effect of growth of the world economy (largely induced by technolgy and globalization) is less clear and highly debated: the benefits of rising incomes and aggregate GDP growth rates associated with gloabalization have not been shared equally across all segements of the population. Indeed, income inequality has risen in most countries and regions over the past two decades, including in developed countries which were thought to have reached levels of prosperity where inequality would level off in line with the Kuznets hypothesis. Since this period has also been associated with unprecedented trade and more recently financial integration, much of the debate over rising inequality has focused on the role that globalization - especially of trade - has played in explaining inequality patterns.\\
Understanding the causes of inequality is fundamental to devising policy measures that can allow the rising prosperity of recent decades to be shared more broadly than has been evident so far. Reducing inequaltiy helps to:
\begin{itemize}
\item[$\hookrightarrow$] achieve a more egalitarian distribution of income and addressing welfare and social concerns that widening disparities in income raise
\item[$\hookrightarrow$] overcome the lack of economic opportunity that occurs when not all economic agents are able to fully exploit the new opportunities created by globalization $\rightarrow$ increasing inequality may limit the growth potential of the economy by not matching capital and labor as efficiently as possible
\item[$\hookrightarrow$] make a greater proportion of the population less vulnerable to to poverty (when economic shocks of various kinds occur)
\end{itemize}
%
%
%
%
%
\clearpage