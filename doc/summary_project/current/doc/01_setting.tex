\section{Model}
\begin{itemize}
\item[\underline{\textbf{\textit{Setting:}}}] we consider a panel of observations with a fixed cross sectional unit $i$ at some time point $t$, and the number of income groups denoted as $M_{it}$ (which may vary over both indices though typically changes only with $i$ i.e. $M_{it}\equiv M_{i}$):
\begin{itemize}
\item $\bs{\underline{n}}_{it}=\left\{\underline{n}_{it}^{(k)}\right\}_{k = 1}^{M_{it}}$:
number of observations in the $M_{it}$ income groups 
\item $\bs{\underline{z}}_{it}=\left\{z_{it}^{(k)}\right\}_{k = 1}^{M_{it}-1}$:
group boundaries with the last boundary $z_{it}^{M_{it}}=\infty$
\item $\bs{\underline{\bar{y}}}_{it}=\left\{\bar{y}_{it}^{(k)}\right\}_{k = 1}^{M_{it}}$:
group mean incomes
\item individual incomes are generated by some parametric income distribution $F_y(y;\Theta)$ with $P$-variate parameter vector $\Theta=(\Theta^{(1)},\ldots,\Theta^{(P)})$. The latter will be modeled componentwise as a latent stochastic processes over time on the logarithmic scale. More precisely, fixing the cross-sectional unit $i$, for $t=1,\ldots,T$, and each component $p=1,\ldots,P$ we have $\log\left(\Theta_t^{(p)}\right)=x^{p}_t$
\item the latent state transitions for each parameter component $p=1,\ldots,P$  typically follow some autoregressive process 
\begin{align*}
\xt = \phi\xtt+\zt\bz+\errX\;,~\errX\sim\NID{0}{\sigma_X^2}\;,
\end{align*}
that includes a set of regressors $\zt$ that describe various effects of globalization such as trade openess, urbanization, level of education etc.
\end{itemize}
%
%
%
%
%
\newpage
\item[\underline{\textbf{\textit{DGP 1:}}}] DGP 1 implies fixed $n_{it}$'s but random group boundaries $z_{it}$'s and group means $\bar{y}_{it}$'s. Let each $\underline{\bar{y}}_{it}\given \underline{z}_{it}^{(k-1)},\underline{z}_{it}^{(k)} \overset{approx.}{\sim} \mathcal{N}_{it}(\bar{y}_k\given z_{it}^{(k)}\Theta)$. Mean incomes are conditionally independent, given regressors $\bs{Z}_{it}$. Assuming individual components of the likelihood functions are approximated via the central limit theorem, the joint likelihood function can be approximately written as
\begin{align*}
L\left(\underline{\Theta};\bs{\underline{\bar{y}}}_{it}\given\bs{Z}_{it}\right)
&=
\prod_{i=1}^N \prod_{t=1}^T 
\prodkM 
\underbrace{\mathcal{N}
\left(\bar{y}_{it}^{(k)} \given z_{it}^{(k-1)},z_{it}^{(k)},\Theta_{it}\right)}_
{\text{Part I}}
\times
\underbrace{
f_z\left(z_{it}^{(k)}\given z_{it}^{(k-1)}, \Theta_{it}\right)
}_{\text{Part II}}
\\
&\times 
\mathcal{N} \left(\bar{y}_{it}^{(1)} \given z_{it}^{(1)},\Theta_{it}\right)
\times 
\mathcal{N} \left(\bar{y}_{it}^{(M_{it})} \given z_{it}^{(M_{it})},\Theta_{it}\right) 
\times 
\underbrace{f_z\left(z_{it}^{(1)}\given \Theta_{it}\right)}_{\text{Part IV}}
\;.
\end{align*}
Part I: $\log\left\{
\mathcal{N}
\left(
\bar{y}_{it}^{(k)} 
\given 
z_{it}^{(k-1)},z_{it}^{(k)},
\Theta_{it}
\right)
\right\}$
\begin{align*}
\log\left\{
\mathcal{N}
\left(
\bar{y}_{it}^{(k)} 
\given 
z_{it}^{(k-1)},z_{it}^{(k)},
\Theta_{it}
\right)
\right\}
&=
\log
\left
\{
\frac{1}{\sqrt{2\pi\sigma^2_k\left(\Theta_{it}\right)}}
\exp
\left\{
-\frac{1}{2}
\frac{\left(\bar{y}_{it}^{(k)}-\mu_k\left(\Theta_{it}\right)\right)^2}{\sigma^2_k\left(\Theta_{it}\right)}
\right\}
\right
\}\\
&=
-\frac{1}{2}
\log\left(2\pi\sigma^2_k\left(\Theta_{it}\right)\right)
-\frac{1}{2}
\frac{\left(\bar{y}_{it}^{(k)}-\mu_k\left(\Theta_{it}\right)\right)^2}{\sigma^2_k\left(\Theta_{it}\right)}\\
&=
-\frac{1}{2}
\left[
\log\left(2\pi\sigma^2_k\left(\Theta_{it}\right)\right)
+
\frac{\left(\bar{y}_{it}^{(k)}-
\mu_k\left(\Theta_{it}\right)\right)^2}
{\sigma^2_k\left(\Theta_{it}\right)}
\right]\;.
\end{align*}
Part II: $\log\left\{
f_z\left(z_{it}^{(k)}\given z_{it}^{(k-1)}, \Theta_{it}\right)
\right\}
$
\begin{align*}
&\log\left\{
f_z\left(z_{it}^{(k)}\given z_{it}^{(k-1)}, \Theta_{it}\right)
\right\}
=\\
&+\log
\left\{
\left(n-n_{k-1}^c\right)!
\right\}
-
\log
\left\{
\left(n_k^c-n_{k-1}^c-1\right)!
\right\}
-
\log
\left\{
\left(n-n_{k}^c\right)!
\right\}\\
&+\left(n-n_{k}^c\right)
\log\left\{1-F_y\left(z_{it}^{(k)};\Theta_{it}\right)
\right\}
-
\left(n-n_{k-1}^c\right)
\log\left\{1-F_y\left(z_{it}^{(k-1)};\Theta_{it}\right)
\right\}\\
&+
\left(n_k^c-n_{k-1}^c-1\right)
\log
\left\{
F_y\left(z_{it}^{(k)};\Theta_{it}\right)-F_y\left(z_{it}^{(k-1)};\Theta_{it}\right)
\right\}\\
&+
\log\left\{f_y\left(z_{it}^{(k)};\Theta_{it}\right)\right\}\;.
\end{align*}
Part IV: $\log\left\{f_z\left(z_{it}^{(1)}\given \Theta_{it}\right)\right\}$
\begin{align*}
&\log\left\{
f_z\left(z_{it}^{(1)}\given \Theta_{it}\right)
\right\}
=\\
&+\log\left\{n!\right\}-\log\left\{\left(n_1^c-1\right)!\left(n-n_1^c\right)!\right\}\\
&+\left(n_1^c-1\right)\log\left\{F_y\left(z_{it}^{(1)};\Theta_{it}\right)\right\}
+\left(n-n_1^c\right)\log\left\{1-F_y\left(z_{it}^{(1)};\Theta_{it}\right)\right\}\\
&+
\log\left\{f_y\left(z_{it}^{(1)};\Theta_{it}\right)\right\}\;.
\end{align*}
%
%
%
%
%
\newpage
One example of for an individual income distribution is the four parameter GB2:
\begin{align*}
F_{GBP}(y;a,b,p,q)=B(d;p,q)=\frac{\int_{0}^d t^{p-1}(1-t)^{q-1}dt}{\text{B}(p,q)}\;,~d=\frac{(y/b)^{a}}{1+(y/b)^{a}}\;,
\end{align*}
where each parameter is strictly positive  $a,b,p,q>0$.
%
%
%
%
%
\newpage
\item[\underline{\textbf{\textit{DGP 2:}}}] DGP 2 implies fixed group boundaries $z_{it}$'s but random $n_{it}$'s and $\bar{y}_{it}$'s. We consider only the likelihood related to the number of observations in each group i.e. the classical likelihood framework following McDonald (1984).\\
Let each $\bs{\underline{n}}_{it}\overset{ind.}{\sim} \text{MNL}_{it}\left(\pi_{it}^{(1)},\ldots,\pi_{it}^{(M_{it})}\right)$ conditional on regressors $\bs{Z}_{it}$ s.th. 
the joint likelihood function is
\begin{align*}
L\left(\Theta;\bs{\underline{n}}_{it}\given\bs{Z}_{it}\right)=\prod_{i=1}^N \prod_{t=1}^T \text{MNL}_{it}(\pi_{it}^{(1)},\ldots,\pi_{it}^{(M_{it})})\;.
\end{align*}
The probability for an individual income to occur in income group $k=1,\ldots,M_{it}$ and is given as $\pi_{it}^{(k)}=\left(F_y(z_{it}^{(k)};\Theta)-F_y(z_{it}^{(k-1)};\Theta)\right)$. One example of for an individual income distribution is the four parameter GB2:
\begin{align*}
F_{GBP}(y;a,b,p,q)=B(d;p,q)=\frac{\int_{0}^d t^{p-1}(1-t)^{q-1}dt}{\text{B}(p,q)}\;,~d=\frac{(y/b)^{a}}{1+(y/b)^{a}}\;,
\end{align*}
where each parameter is strictly positive  $a,b,p,q>0$.
\end{itemize}
