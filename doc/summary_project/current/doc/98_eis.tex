PEIS: standard SMC constructed from the output of an EIS algorithm:
\begin{itemize}
\item[$\Rightarrow :$]replace resampling and propagation scheme by a scheme that favours particles that are more likely to survive the next resampling step
\item[$\Rightarrow :$]this is achieved by augmenting  standard targets $\targT{}$ to include future $y_t$-measurements $\rightarrow$ forward looking 
\item[$\Rightarrow :$]these extended auxiliary densities are then targeted by a SMC based on an EIS proposals and appropriately adjusted weights (that take into account the discrepancy between the new auxiliary targets and the particular EIS proposals)
\end{itemize}
\begin{itemize}
\item The auxiliary targets are:
\begin{align*}
\targt{} 
\pt
\kernt{} 
\equiv
\p\left(x_\seqt, y_\seqt\right) \propNORM{t+1}{t}{}
= \p\left(x_\seqt, y_\seqt\right) \p\left(y_{(t+1):T}\given x_{t}\right)\;,
\end{align*} 
where the last equality holds for the conceptual globally fully adapted PEIS where $\propNORM{t+1}{t}{}=\p\left(y_{(t+1):T}\given x_{t}\right)$ together with the corresponding optimal proposals sets the global IS ratio equal to zero. However, these are typically not feasible and an approximation to these optimal proposals and optimal normalizing constant is obtained via auxiliary EIS regressions.
\item The proposal is: 
\begin{align*}
\propt{}= \propAUXt{} = \frac{\propKERN{} }{\propNORM{t}{t-1}{}}\;.
\end{align*}
\item The resulting weights are:
\begin{align*}
\wtun
&= 
\wtt
\frac
{\kernt{i}}
{\gamma_{t-1}\left(x_{1:(t-1)}^{i}\right)\propt{i}}\\
&=
\wtt 
\frac{\p\left(x_\seqt^i, y_\seqt^i\right)\propNORM{t+1}{t}{i}
\propNORM{t}{t-1}{i}}
{\p\left(x_{1:(t-1)}^i, y_{1:(t-1)}^i\right)\propNORM{t}{t-1}{i}
\propKERN{i}} \\
&=
\wtt \frac{\p\left(x_t^i, y_t^i \given x_{1:(t-1)}^i, y_{1:(t-1)}\right)
\propNORM{t+1}{t}{i}}
{\propKERN{i}} \\
&=
\wtt \frac{g_{\rt}\left(y_t \given x_t^i\right) f_{\rt}\left(x_t^i \given x_{t-1}^i\right)
\propNORM{t+1}{t}{i}}
{\propKERN{i}}\;.
\end{align*}
\end{itemize}
\newpage
\section*{GZ model}
Derivation of SMC-EIS weights when resampling is performed in every period and for one exemplary state trajectory $x_t$ e.g. $x_{a,t,n}\equiv x_t$ for some $n=1,\ldots,N$, $t=1,\ldots,T$.
\begin{align*}
\wt &= 
\frac
{
g_{\rt}\left(y_t \given x_t^i\right) f_{\rt}\left(x_t^i \given x_{t-1}^i\right)
\propNORM{t+1}{t}{i}
}
{
\propKERN{i}
}
\;.\\
\log\left(\wt\right) 
&=
\log\left(g_{\rt}\left(y_t \given x_t^i\right)\right) +
\log\left(f_{\rt}\left(x_t^i \given x_{t-1}^i\right)\right) +
\log\left(\propNORM{t+1}{t}{i}\right) -
\log\left(\propKERN{i}\right)\;.
\end{align*}
Deriving each part of the log-weights:
\begin{itemize}
\item[1.] 
\begin{align*}
g_{\rt} \left(y_t \given x_t^i\right) 
&=
\frac{K!}{y_{t,1}!\cdot y_{t,K}!}
\prodk \pi_k^{y_{t,k}}
\\
\log\left(g_{\rt}\left(y_t \given x_t^i \right)\right)
&=
\log\left(K!\right) -
\sumk \log\left(y_{t,k}!\right) +
\sumk \log\left(y_{t,k}\right)\log\left(\pi_k\right)
\\
\log\left(g_{\rt}\left(y_t \given x_t^i \right)\right)
&\pt \sumk \log\left(y_{t,k}\right)\log\left(\pi_k\right)\;.
\end{align*}
\item[2.]
\begin{align*}
f_{\rt}\left(x_t^i \given x_{t-1}^i\right)
&=
\frac{1}{\sqrt{2\pi\sigXa^2}}
\exp
\left\{
-\frac{1}{2}
\frac{\left(x_t^i-\mu_{x_a}\right)^2}
{\sigXa^2}
\right\}
\\
\log\left(f_{\rt}\left(x_t^i \given x_{t-1}^i\right)\right)
&= 
-\frac{1}{2}\log\left(2\pi\sigXa^2\right)
-\frac{1}{2}\frac{\left(x_t^i-\mu_{x}\right)^2}
{\sigXa^2}\;,
\end{align*}
where $\mu_{x}=\bs{z}_a\bs{\beta}_{z_a}^*$ if e.g.  $x_t\equiv x_{t,a}=\phi x_{t-1,a} + z_a\bs{\beta}_{z_a} + \varepsilon_{x_a}$ and $\bs{\beta}_{z_a}^*=\left(\phi,\bs{\beta}_{z_a}\right)$.
\item[3.]
\begin{align*}
\propNORM{t+1}{t}{i}
&=
\\
\log\left(\propNORM{t+1}{t}{i}\right)
&=
\end{align*}
\item[4.]
\begin{align*}
\propKERN{i}
&=
\\
\log\left(\propKERN{i}\right)
&=
\end{align*}
\end{itemize}
\newpage
Derivation of AS weights for PEIS when resampling is performed in every period and for one exemplary state trajectory $x_t$ e.g. $x_{a,t,n}\equiv x_t$ for some $n=1,\ldots,N$, $t=1,\ldots,T$.
\begin{itemize}
\item In general, the AS-weights are computed as
\begin{align*}
\asw 
\pt 
\wtt
\times 
\frac
{
\gamma_T
\left(
x_{1:(t-1)}^i, x^{\mathcal{R}}_{t:T}
\right)
}
{
\gamma_{t-1}
\left(
x_{1:(t-1)}^i
\right)
}\;.
\end{align*}
\item The globally fully adapted SMC targeted by PEIS uses targets 
$\targt{} 
\pt
\p\left(x_\seqt, y_\seqt\right) \p\left(y_{(t+1):T}\given x_{t}\right)$ and
$q_t\left(x_t \given x_{1:(t-1)}\right)
\equiv
\p\left(x_t \given x_{t-1}, y_{t:T}\right)$. In this setting the AS weights become
\begin{align*}
\asw 
&\pt 
1 
\times 
\frac
{
\gamma_T
\left(
x_{1:(t-1)}^i, x^{\mathcal{R}}_{t:T}
\right)
}
{
\gamma_{t-1}
\left(
x_{1:(t-1)}^i
\right)
}
=
1
\times
\frac
{ 
\p\left(
y_{t:T},x_{t:T}^{\mathcal{R}}\given x_{t-1}^i
\right)
}
{
\p\left(
y_{t:T}\given x_{t-1}^i
\right)
}
\\
&=
1
\times 
\p
\left(
x_t^{\mathcal{R}} \given x_{t-1}^i, y_{t:T}
\right)\;.
\end{align*}
Under PEIS, which is designed to provide the best possible approximation to the globally fully adapted SMC, the constant prior weights, $\pt 1$, are replaced by $\wtt$ and the optimal EIS density by the PEIS density (which is obtained by solving the least squares optimization problem before propagating particles from $t=1,\ldots,T$) and given as
\begin{align*}
\propt{}= \propAUXt{} = \frac{\propKERN{} }{\propNORM{t}{t-1}{}}\;.
\end{align*}
\end{itemize}