\section{Rising Income Inequality: Technology, or Trade and Financial Globalization (F. Jaumotte, S. Lall, and C. Papageorgiou - 2013)}
\subsection*{TO-DO}
\begin{itemize}
\item[I.] welche Regressoren, welche Zeitpunkte, welche Länder. Im Detail:
\begin{todolist}
\item welche Länder sind verfügbar
\item welche Regressorn sind für diese Länder verfügbar
\item welche Zeitpunkte sind verfügbar
\item gibt es Regionen für die verfügbaren Länder
\item[\done] 
\item[\wontfix] profit
\end{todolist}
\item[II.]  gibt es für entsprechende Länder auf povcal auch income Daten also welcher DGP: DGP1 oder DGP2 und wie viele Klassen der entsprechende DGP hat
\begin{todolist}
\item welche income Daten sprich DGP gibt es
\item was ist denn so die Anzahl an Klassen im Schnitt
\item wie wird das Einkommen gemessen: ist das pro Kopf oder Haushaltseinkommen?
\item 
\end{todolist}
\item[III.] wie groß ist das missing data problem: nur so ein paar oder mal 5 jahre missing
\item[IV.] wie groß ist das räumliche potential, gibt es cluster wie zB Afrika oder Asien und wie viele Länder pro Region da sind; nachbarschaftsmatrizen basierend auf kontinenten, oder wirtschaftsregionen oder benachbarte grenzen (letzteres eher unwahrscheinlich)
\item[V.] irgendein witziges empirisches finding: räumliche Abhängigkeit oderso bspw.
\item[VI.] warum soll ein Anwender unseren scheiß machen? warum soll er nicht einfach mal die a,b,p,q schätzen und damit seine regressionen machen, sondern diese mehrstufige geschichte von uns. Antworten:
\begin{itemize}
\item Missing data
\item unterschiedliche DGPs?
\item das ist so wie garch vs. stochastic volatility
\item so einem Argument wird man nicht gut begegnen können
\end{itemize}
\item reparametrisierung der GB2: ein mittelwertparameter und drei shape parameter, vielleicht gibt es ja noch varianz oder schiefe sodass die parameter eine bedeutung bekommen: marginale effekte auf Gini, head-count-ratio, theyl index und lorenz kurve: hast du einkommensverteilung hast du alles! Gutes Argument für unser papier.
\end{itemize}
\subsection*{Data}
Newly compiled panel of 51 countries over a 23-year period from 1981 to 2003:
\subsubsection*{Regressors}
1. TRADE GLOBALIZATION
\begin{itemize}
\item[1.] de facto trade openess := sum of exports and imports of non-oil goods and services
\item[2.] 
\end{itemize}

%\begin{itemize}
%\item subcomponents of trade and financial globalization e.g. exports of manufacturing vs. agriculture, and portfolio debt and equity flows vs FDI
%\end{itemize}
